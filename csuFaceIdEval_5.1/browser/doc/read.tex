%% LyX 1.3 created this file.  For more info, see http://www.lyx.org/.
%% Do not edit unless you really know what you are doing.
\documentclass[english]{article}
\usepackage[T1]{fontenc}
\usepackage[latin1]{inputenc}
\usepackage{pslatex}

\makeatletter
%%%%%%%%%%%%%%%%%%%%%%%%%%%%%% Textclass specific LaTeX commands.
 \newenvironment{lyxcode}
   {\begin{list}{}{
     \setlength{\rightmargin}{\leftmargin}
     \setlength{\listparindent}{0pt}% needed for AMS classes
     \raggedright
     \setlength{\itemsep}{0pt}
     \setlength{\parsep}{0pt}
     \normalfont\ttfamily}%
    \item[]}
   {\end{list}}

%%%%%%%%%%%%%%%%%%%%%%%%%%%%%% User specified LaTeX commands.
\usepackage[T1]{fontenc}
\usepackage[latin1]{inputenc}

\makeatletter
\usepackage[T1]{fontenc}
\usepackage[latin1]{inputenc}

\makeatletter
\usepackage[T1]{fontenc}
\usepackage[latin1]{inputenc}
\usepackage{geometry}
\geometry{verbose,letterpaper,tmargin=1in,bmargin=1in,lmargin=1in,rmargin=1in}

\makeatletter
\usepackage[T1]{fontenc}
\usepackage[latin1]{inputenc}
\usepackage{graphicx}

\makeatletter
\usepackage[T1]{fontenc}
\usepackage[latin1]{inputenc}

\makeatletter
\usepackage[T1]{fontenc}
\usepackage[latin1]{inputenc}

\makeatletter
\usepackage[T1]{fontenc}
\usepackage[latin1]{inputenc}
\usepackage{graphicx}

\makeatletter
\usepackage[T1]{fontenc}
\usepackage[latin1]{inputenc}
\usepackage{graphicx}
\usepackage{pslatex}

\makeatletter
\usepackage{babel}
\makeatother


\usepackage{babel}
\makeatother

\usepackage{babel}
\makeatother

\usepackage{babel}
\makeatother

\usepackage{babel}
\makeatother

\usepackage{babel}
\makeatother

\usepackage{babel}
\makeatother

\usepackage{babel}
\makeatother

\usepackage{babel}
\makeatother
\begin{document}

\title{The FERET Browser User's Guide}

\maketitle
\begin{abstract}
The Feret Browser Graphical User Interface system aids in selecting
the person specific and image specific covariates for each of the
images present in the database. The person specific covariates need to be selected 
just once for each of the subjects in the database. Whereas the image specific 
covariates need to be selected for each of the images.
\end{abstract}

\section{Installation:}


\subsection{Creating the Binaries:}

The Feret Browser system is written in QT, you need to have the standard
QT libraries installed on the host system. Use the following to compile
and run the browser.

\begin{lyxcode}
qmake~-project

qmake

make~
\end{lyxcode}

\section{System Overview:}


\subsection{Command Line arguments:}

The application takes three command line arguments.


\paragraph{\textmd{First argument - specifies the folder containing the original
Feret Images.}}


\paragraph{\textmd{Second argument - specifies the folder containing the normalized
Images.}}


\paragraph{\textmd{Third argument - This is an optional argument. If suppose
there exits a file 'A' with the details of the covariates of say ten
subjects. Now details of subjects other then thoes ten need to be
appended to the file 'A'. Then the Feret Browser needs to be opend
with this third argument in addition to the first and second argument.
in this case the third argument needs to specify the path where file
'A' exists.}}


\subsection{When the Feret Browser loads up, the application window looks as
follows:\protect \protect \protect \\
 }

\includegraphics[%
  scale=0.5]{/s/chopin/a/grad/charu/vision/csuFaceIdEval.4.0/browser/doc/snap3.jpg}


\paragraph{\textmd{At the upper left corner of the window the original image
of the person is displayed. }}


\paragraph{\textmd{Besides the original image the normalized image of the person
loads up which is smaller in size than the original image. }}


\paragraph{\textmd{Below the images a table is seen which gives details about
both the images. }}


\paragraph{\textmd{On the right side of the images, radio buttons are seen which
can be used to select the person and image specific covariates. }}

\clearpage
\subsection{Person specific covariates: }


\subparagraph{Are attributes that define the person in general and have the same
value for all the images of the person in the database.}

\begin{itemize}
\item Gender 
\item Race 
\item Age 
\item Skin 
\end{itemize}

\subsection{Image specific covariates:}


\subparagraph{Are attributes that define the person in the image and may vary for
the images of the person in the database.}

\begin{itemize}
\item Glasses 
\item Facial Hair 
\item Expression 
\item Mouth 
\item Eyes 
\item Bangs 
\item Makeup 
\end{itemize}

\subsubsection*{\textmd{Below the Radio buttons there are five buttons, which are
used to load, store and clear the covariates data selected by the
Radio buttons as their name indicates. The keyboard Grab is for disabling
and enabling the keyboard short-cuts and the last button is for closing
and exiting the Feret Browser.}}


\section{Detailed operation of the System:}

\begin{enumerate}
\item Choose any image by clicking on the image name displayed in the table
below the images. 
\item On doing so the original image shows up along with the normalized
image. On the basis of the original image, the image covariates need
to be selected. The person specific covariates once selected remains
the same for all the images of the same subject. 
\item Every image that is selected before the Feret Browser is closed is
marked by a 'YES' in the last column named ' DATA' of the table. The
' YES ' in this coloum indicates that the covariates for thoes images
have been selected.
\end{enumerate}

\subsection{Saving Eye Co-ordinates}

\subparagraph{The Eye co-ordinates can be produced by left clicking on the left eye and by right clicking on the right eys. On do so a red cross appears on the left and right eye and the 'x' and 'y' co-ordinates of this position is saved into a file named EyeCoords.txt. In this file each row contains a unique image name followed by the 'x'and 'y' co-ordinates of the left and right eyein that order. }
\subsection{SaveData Button}


\subparagraph{\textmd{When this button is clicked a dialog box props up in which
the name of a file needs to be specified. This operation writes to
that file a 'number' which, indicates the number of images whoes details
will get saved to the same file. Following this number, details of
all the images marked with a 'YES' in the 'DATA' coloum of the table
are saved into the file. The details include the person and image
specific covariates. In addition to this the file contains the values
set for two variables for each of the images saved. The first being
'AGE2'. This variable takes values of either 'YOUNG' or 'OLD'. The
variable takes the value 'YOUNG', if the person in the image has his
'AGE' (person specific covariate) set to 30 or below 30. Else 'AGE2'
is set to 'OLD'. The second varaiable is 'SKIN2'. It also takes 2
values 'CLEAR' or 'OTHER'. It is set to 'CLEAR' if the 'SKIN' (person
specific covariate) is set to 'CLEAR' else it is set to 'OTHER'.}}


\subsection{LoadData Button }


\subparagraph{\textmd{When this button is clicked a dialog box props up and lets
the user choose the file ( this file should be the one created by
the SaveData button or along the similar format) to be loaded. The
LoadData button can be used several times to load different files.
Loading the second file does not erase the covariate values set for images by loading the 
first file.This is one way by which covariate values for different images from several
files can be combined.  }}


\subsection{ClearData Button}


\subparagraph{\textmd{This button aids to clear all the covariate selection made
for all the images choosen before the click of this button. Also the
'YES' value set in the Last 'DATA' coloum of the table is now set
to 'NO'. The click of the ClearData button resets the covariates to
their default value, and helps to start the selection afresh.}}


\subsection{Keyboard Grab}


\subparagraph{\textmd{There are some keyboard short-cuts for selecting only the
person specific covariates. This is by default disabled. To enable
these short-cuts this button needs to be used. This is a toggle button.
On enabling the Keyboard Grab, the Feret Browser revives all the key
board shortcuts and any other application will not get any keyboard
events until the Keyboarb Grab has been deactivated.}}


\subsubsection{The Keyboard Short-cuts are as follows:}


\paragraph{Gender : }


\subparagraph{\textmd{key 1 -Male}}


\subparagraph{\textmd{key 2 - Female}}


\paragraph{Race :}


\subparagraph{\textmd{key q - White}}


\subparagraph{\textmd{key w - Black}}


\subparagraph{\textmd{key e - Asian}}


\subparagraph{\textmd{key r - Other}}


\paragraph{Age:}


\subparagraph{\textmd{key a - Teen}}


\subparagraph{\textmd{key s - 20}}


\subparagraph{\textmd{key d - 30}}


\subparagraph{\textmd{key f - 40}}


\subparagraph{\textmd{key g - 50}}


\subparagraph{\textmd{key h - 60+}}


\paragraph{Skin:}


\subparagraph{\textmd{key z - Clear}}


\subparagraph{\textmd{key x - Freckled}}


\subparagraph{\textmd{key c - Wrinkled}}


\subparagraph{\textmd{key v - Both}}


\subparagraph{\textmd{key b - Other}}


\subparagraph{\textmd{The lower or upper case of the keys have the same effect
when the Keyboard Grab is activated.}}


\subsection{Quit Button}


\subparagraph{\textmd{The Feret Browser can be closed and exited gracefully on
clicking the Quit button.}}
\end{document}
